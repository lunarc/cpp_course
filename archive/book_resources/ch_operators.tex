\section{ch\_operators}\label{chux5foperators}

\begin{Shaded}
\begin{Highlighting}[]
\OtherTok{#include <iostream>}

\KeywordTok{using} \KeywordTok{namespace} \NormalTok{std;}

\DataTypeTok{int} \NormalTok{main()}
\NormalTok{\{}
    \DataTypeTok{int} \NormalTok{a, b;}
    
    \NormalTok{a = }\DecValTok{42}\NormalTok{;}
    \NormalTok{b = }\DecValTok{26}\NormalTok{;}
    
    \NormalTok{a += b; }\CommentTok{// a = a + b}
    
    \NormalTok{cout << }\StringTok{"a = "} \NormalTok{<< a << endl;}
\NormalTok{\}}
\end{Highlighting}
\end{Shaded}

\begin{Shaded}
\begin{Highlighting}[]
\OtherTok{#include <iostream>}

\KeywordTok{using} \KeywordTok{namespace} \NormalTok{std;}

\DataTypeTok{int} \NormalTok{main()}
\NormalTok{\{}
    \DataTypeTok{int} \NormalTok{a, b, c;}
    
    \NormalTok{a = }\DecValTok{42}\NormalTok{;}
    
    \NormalTok{b = ++a;}
    \NormalTok{c = a++;}
    
    \NormalTok{cout << }\StringTok{"b = "} \NormalTok{<< b << endl;}
    \NormalTok{cout << }\StringTok{"c = "} \NormalTok{<< c << endl;}
\NormalTok{\}}
\end{Highlighting}
\end{Shaded}

\begin{Shaded}
\begin{Highlighting}[]
\OtherTok{#include <iostream>}

\KeywordTok{using} \KeywordTok{namespace} \NormalTok{std;}

\DataTypeTok{int} \NormalTok{main()}
\NormalTok{\{}
    \DataTypeTok{int} \NormalTok{number;}
    
    \NormalTok{cout << }\StringTok{"Enter a number : "}\NormalTok{;}
    
    \NormalTok{cin >> number;}
    
    \DataTypeTok{int} \NormalTok{outValue = (number>}\DecValTok{50}\NormalTok{) ? }\DecValTok{42} \NormalTok{: }\DecValTok{21}\NormalTok{;}
    
    \NormalTok{cout << }\StringTok{"outValue = "} \NormalTok{<< outValue;}
\NormalTok{\}}
\end{Highlighting}
\end{Shaded}

